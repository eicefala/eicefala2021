\chapter{The Behavior of High Vowels in Unstressed Final Syllables in Fortalezenses' Informal Speech}\label{ch:franciscoaler5}
\chapterauthor[1]{Francisco Alerrandro da Silva Araujo}
\chapterauthor[1]{Ronaldo Mangueira Lima Júnior}
\begin{affils}
\chapteraffil[1]{Universidade Estadual de Campinas}
\end{affils}

The goal of this study is to analyze the production of the high vowels /i/ and /u/ in final post-tonic syllables in the spontaneous speech of Fortaleza. We wish to investigate the phonetic variation in the realization of these vowels as a possible indicator of linguistic change in this variety of Brazilian Portuguese. Data used for the analysis were taken from the NORPOFOR - Oral Norm of Popular Portuguese of Fortaleza-CE database. The independent variables investigated were: age group, educational level, gender, the stressed vowel, the preceding consonant, and word frequency. In 16 recordings of the DID type (Dialogue between Informant and Documenter), words ending in syllable-final unstressed /i/ and /u/ were inspected. We use Exemplar Theory (JOHNSON, 1997; PIERREHUMBERT, 2001; BYBEE, 2001; CRISTÓFARO SILVA; GOMES, 2004) as the theoretical basis for the analysis and for the interpretation of the data. It is one of the premises of this theory that experience impacts mental representations, which are defined probabilistically over all instances of the category attested in language use. This study was also based on Acoustic-Articulatory Phonology (BROWMAN \& GOLDSTEIN, 1989; ALBANO, 2001), investigating the possible influences that consonants can have on the realization of the high vowels under study. The acoustic analysis of the data were done with the PRAAT program, while the quantitative analysis were carried out in R. We found a high rate of vowel deletion in the data, especially regarding the front vowel /i/, while the mixed-effects logistic model revealed that the only significant independent variables are those related to preceding vowel (manner of articulation and voicing), with fricatives, affricates, and plosives, especially the voiceless ones, favoring the deletion of the vowels evaluated.
