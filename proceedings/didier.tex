\chapter{The multiple dimensions of speech: old questions and new challenges}\label{ch:didier}
\chapterauthor[1]{Didier Demolin}
\begin{affils}
\chapteraffil[1]{Laboratoire de Phonétique et Phonologie\\CNRS-UMR 7018}
\end{affils}

\noindent 
Human speech, a product of the evolution of primates, can in essence be defined
in terms of a signal. This is an acoustic wave varying over time with amplitude
and frequency modulations, due to the articulatory movements of the vocal
tract’s organs. To perform these movements, motor controls are required, whose
interactions with the aerodynamic parameters produce the acoustic signal. The
main objective of research in this domain is to understand which primary
principles,biological, physical and cognitive, to be based on to explain
the production and perception of speech in the world’s languages and to make the
fundamental question: \textit{how does it work?}

Among the main fields of activity involved in the study of sounds and sound
systems of languages are the engineering sciences with the dimensions of
automatic processing (speech recognition and synthesis);phonetics and phonology
(the linguistic aspects); and pathological aspects(how to explain what doesn't
work anymore or less well). This includes knowledge of similar fundamental
principles. To these dimensions a readded physics, biology, cognition and
neuroscience. These fields involves in-depth knowledge of various
interconnected fields to explain how sounds and sound systems work. Therefore in
addition to the symbolic dimension, anatomical, physiological, acoustic,
aerodynamic, articulatory,auditory, proprioceptive, historical (phylogenies
and diachrony),ecological, temporal, dynamic and self-organized aspects can,
and should,be integrated in the explanation of the studied phenomena.

The complexity and interactions of these dimensions find new light in the
paradigms resulting from the study of complex systems, which makes it possible
to address old issues again, such as the search for a possible speech code,
invariants and primitives. From these issues, others arise, such as the
understanding of the open or closed nature of sound systems, which is far from
being resolved. Explaining the diversity,complexity and dynamics of sound
systems involves understanding the nature of variation in speech phenomena. How
can we show that spontaneous speech, laboratory speech and pathological aspects
are based on the same principles?

The evolution of theory, models, new statistical tools,computational, big data
and deep learning tools, allow these issues to be addressed in a new light. New
measuring instruments such as real-time magnetic resonance imaging, functional
magnetic resonance, three-dimensional or four-dimensional ultrasound, digital
endoscopy,electroencephalography (EEG) and many other recent tools make
it possible to accurately observe, measure and quantify speech phenomena as well
as bring to discussion fundamental issues still unresolved or poorly understood. 

The lecture will discuss the controlled and automatic aspects involved in the
control of breathing in speech, issues in speech embodiment, the quantal
aspects of speech, the importance of thresholds values in aerodynamic and
acoustic parameters, types of feedback (acoustic and proprioceptive) in speech
phenomena and new ways to explain and formalize the source, the initiation and
propagation of sound changes. This last point by using and adapting
population ecology models to speech. 


%\bibliographystyle{plainnat}
%\bibliography{didier.bib}
