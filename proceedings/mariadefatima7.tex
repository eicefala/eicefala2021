\chapter{Comparing singing and speech voice of twins: a dialogue between Psycholinguistics and Musical Studies}\label{ch:mariadefátima7}
\chapterauthor[1]{Maria de Fátima de Almeida Baia}
\chapterauthor[1]{Waldemar Ferreira Netto}
\chapterauthor[1]{Laís Rodrigues Silva}
\chapterauthor[1]{Ana Cristina Oliveira Santos Bockorni}
\begin{affils}
\chapteraffil[1]{Universidade Estadual do Sudoeste da Bahia}
\end{affils}

In this study, we establish a dialogue between the early prosodic and musical development studies, highlighting the lack of dialogue among the areas of Psycholinguistics, Music and related areas. Prosodic development studies point out to universals in early intonational patterns (SCARPA, 1997) as the alternation between high and low tones in intonation, short melodies, binary rhythmic sequence of (un)stressed syllables, etc, whereas studies on perception and musical production of babies find specific musical intervals of each stage of development (PARIZZI, 2006). However, psycholinguistics studies, in general, observe linguistic aspects in early production/perception (SCARPA, 1997; DEMUTH, 1996; JUCSZYK, 1997) whereas studies on early musical development claim there would be “pure” musical aspects only at the beginning of development as language aspects would appear by the first year. In addition to presenting a debate, we investigated, through the comparison of the mid tone in singing and speech voice, if there are differences between the two modalities in the developing voice of twins at the age 1 to 2 years. After analyzing the data, we obtained approximate means of mid tone in speech and singing voice, although the standard deviation values showed a sort of “total tuning / delimitation” in the children's singing voice. The mean values and their respective dispersion as the dispersion values among the subjects were analyzed. The χ2 contingency statistic pointed to a significant difference only in terms of dispersion between subjects (P<0.01); despite a relatively low result (P=0.06), the other values compared still do not allow us to offer advanced conclusions.
