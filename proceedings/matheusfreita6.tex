\chapter{The influence of non-native orthographic patterns in the variable occurrence of word-final [ɪ] in Brazilian Portuguese}\label{ch:matheusfreita6}
\chapterauthor[1]{Matheus Freitas Gomes}
\begin{affils}
\chapteraffil[1]{Universidade Federal de Minas Gerais, Laboratório de Fonologia}
\end{affils}

In this presentation, preliminary results from an exploratory experiment that is part of an ongoing doctoral research are presented. It investigates the weakening and loss of the word-final vowel [ɪ] in Brazilian Portuguese (henceforth BP) considering its implementation in words with two different orthographic patterns. The first set of analysed words comprises native words, that always present an orthographic final <E> vowel, as in leque ‘hand fan’ > [ˈlɛkɪ] ~ [ˈlɛk]. The second set of analysed words comprises borrowings that also present an orthographic final <E> vowel, as in make ‘make-up’ > [ˈmeɪkɪ] ~ [ˈmeɪk]. These two sets of words are related to an expected orthographic pattern in Portuguese, i.e. a final <E>. The third set of analysed words, on the other hand, comprises loanwords that end with an orthographic consonant, as in drink ‘drink’ > [ˈdɾĩkɪ] ~ [ˈdɾĩk], which correspond to a non-native orthographic pattern in Portuguese, i.e. an orthographic final consonant. The weakening and loss of word-final [ɪ] in BP is a variable phenomenon that has been reported in different works, and may affect native and borrowed words (Pagel 1993; Viegas \& Oliveira 2008; Rolo \& Mota 2012; Dias \& Seara 2013; Vieira \& Cristófaro Silva 2015; Meneses \& Albano 2015; Assis 2017). Moreover, there is evidence that orthographic forms might influence speech production and perception (Taft 1982; 2006; Mattingly 1992; Chevrot 1999; D’Andrade \& Rodrigues 1999; Vendelin \& Peperkamp 2006; Baroni 2016), especially in loanword adaptation (Smith 2006; Hamann \& Colombo 2017) and in phonological variation cases (Purse 2019). It was posited the hypothesis that final vowel weakening and loss affect words differently accordingly to their orthographic form. It was predicted that vowel weakening and loss would occur at higher rates in words that present the non-native orthographic pattern, in which the final vowel does not have an orthographic correspondent. In order to test this hypothesis, a picture naming experiment was carried out. Results showed that, for loanwords, vowel absence rates are higher in words with an orthographic final consonant (77.8\%) than in words with an orthographic final <E> (51.2\%). For native words, which always have the orthographic final vowel, the vowel was not present in 67.3\% of the observations. Moreover, when a vowel was produced, it was shorter in loanwords than in native words. Results were analysed in light of Exemplar Models (Johnson 1997; Pierrehumbert 2001; Bybee 2001) and of the Integration of Multiple Patterns model for spelling representations (Treiman \& Kessler 2014), with aims to contribute with the debate on how orthography might play a role in the productions of ongoing phonological phenomena.

\vspace{1ex}

Assis, A. (2017). A emergência de consoantes finais no português brasileiro na microrregião de Araguaína/Tocantins. Universidade Federal de Minas Gerais, Belo Horizonte.

Baroni, A. (2016). Constraint interaction and writing systems typology. Dossiers d’HEL, SHESL, 2016, Écriture(s) et représentations du langage et des langues, 9, 290-303.

Bybee, J. (2001). Phonology and language use. Cambridge: Cambridge University Press.

Chevrot, J. (1999). L’effet Buben: de la linguistique diachronique à l’approche cognitive (et retour). Langue française, 129, 104-125.

Dias, E.; \& Seara, I. (2013). Redução e apagamento de vogais átonas finais na fala de crianças e adultos de Florianópolis: uma análise acústica. Letrônica, 6(1), 71-93.

Hamann, S.; \& Colombo, I. (2017) A formal of the interaction of orthography and perception: English intervocalic consonants borrowed into Italian. Natural Language \& Linguistic Theory, 35, 2017, 683-714

Johnson, K. (1997a). Speech perception without speaker normalization: An exemplar model. In K. Johnson, \& J. Mullenix (eds). Talker Variability in Speech Processing (pp. 145-165). San Diego: Academic Press.

Mattingly, I. (1992). Linguistic awareness and orthographic form. Advances in Psychology, 94(1), 11-26.

Meneses, F.; \& Albano, E. (2015). From Reduction to Apocope: Final Poststressed Vowel Devoicing in Brazilian Portuguese. Phonetica (Basel), 72, 121-137.

Pagel, D. (1993). Contribuição para o estudo das vogais finais inacentuadas em português. Cadernos de Estudos Lingüísticos, 25(1), 67-83.

Pierrehumbert, J. (2001). Exemplar dynamics: word frequency, lenition and contrast. In J. Bybee, \& P. Hooper (eds). Frequency and the emergency of linguistic structure (pp 137-157), Amsterdam: J. Benjamins.

Purse, R. (2019). Variable Word-Final Schwa in French: an OT Analysis. U. Penn Working Papers in Linguistics, 25(1), 199-208.

Rolo, M.; \& Mota, J. (2012). Um estudo sociolinguístico sobre o apagamento de vogais finais em uma localidade rural da Bahia. Signum: Estudos da Linguagem, 15(1), 311-334.

Taft, M. (1982). An alternative to grapheme-phoneme conversion rules. Memory \& Cognition, 10(1), 465-474.

Taft, M. (2006). Orthographically influenced abstract phonological representation: Evidence from nonrhotic speakers. Journal of Psycholinguistic Research, 35(1), 67-78

Treiman, R.; Kessler, B. (2014). How children learn to write words. New York: Oxford University Press.

Vendelin, I.; Peperkamp, S. (2006). The Influence of Orthography on Loanword Adaptation. Lingua, 116(1), 996-1007.

Viegas, M. C; Oliveira, A. J. (2008). Apagamento da vogal em sílaba /l/ V átona final em Itaúna/MG e atuação lexical. Revista da ABRALIN, 2(1), 119-138





