\chapter{“One for all, all for one” -- On process-oriented approaches to multilingual phonetic-phonological development}\label{ch:ubirata}
\chapterauthor[1]{Ubiratã Kickhöfel Alves}
\chapterauthor[1]{Laura Castilhos Schereschewsky}
\begin{affils}
\chapteraffil[1]{UFRGS-CNPq}
\end{affils}

\noindent
Research on language development has shown an interconnection of all the
languages in a multilingual system \citep{herdina2001,pereyron2017}.
This assumption defies a traditional ‘L1 – L2 – L3’ linear order of
influence, as changes in any of the language subsystems may have an
impact on all the others. This also reinforces the importance of a dynamic
approach to language attrition, since the L1 itself shows changes over time.
This is a likely possibility in both dominant and non-dominant L1
environments, as revealed by many studies carried out in the Brazilian
scenario \citep{kupske2015,schereschewsky2018,santos2018,alves2019}.

Aiming to express the interconnectedness of language subsystems over
time, process-oriented approaches \citep{lowie2017,lowie2019} 
have given rise to longitudinal accounts of language development.
Longitudinal methods indicate when changes in each one of the language
subsystems start taking place, revealing how data variability affects
development \citep{chang2021}. In this scenario, many dynamic
approaches to data analyses have been proposed in the last few years
\citep{verspoor2011,hiver2019}. Recent
studies carried out in our research group \citep{albuquerque2019,alves2020,schereschewsky2021} 
suggest that moving correlations \citep{verspoor2011b,bulte2020}, Monte
Carlo techniques \citep{verspoor2011,chang2021}
and Change-Point analyses \citep{taylor2000} may highlight the
modifications that take place in each one of the languages of the
multilingual system, besides suggesting an interconnection among them.

In this presentation, we illustrate these points by presenting two studies
that we have carried out at LABICO (Laboratório de Bilinguismo e Cognição),
at Universidade Federal do Rio Grande do Sul (Brazil). In the first study,
\citet{schereschewsky2021} investigated the development of VOT patterns by
five Brazilian participants (L1: Brazilian Portuguese), learners of English (L2)
and French (L3), all of them residing in their native country. These learners
participated in a three-month longitudinal study, with 12 (weekly) data
points. Following an A-B-A account \citep{hiver2019}, instruction
on English aspiration was provided between weeks 4 and 9, in order to
accelerate variability in the learner’s L2. The results show that an increase
in the aspiration rates in English led to changes in the VOT patterns of both
Brazilian Portuguese (L1) and French (L3).

In the second study, Alves (in progress) has investigated the vowel
development process of an Argentinean (L1: Riverplate Spanish) learner of
English (L2) and Brazilian Portuguese (L3) living in Brazil. This learner was
investigated in a time window of one year and took part in 24 data
collections, which were held every fifteen days. Also following an A-B-A
account \citep{hiver2019}, this participant took part in explicit
instruction classes on the pronunciation of Brazilian Portuguese (vowel and
consonants sounds) between data collections 10 and 15. Our preliminary
results show that changes in F1, F2 and duration in the L3 subsystem have
led to modifications in both the learner’s L1 and L2. Besides, all the vowels
in each subsystem showed dynamic relations with one another, thus
reinforcing the importance of studying all the vowel segments in the
developmental process.

Taken together, the results discussed in this presentation not only provide
empirical evidence to the interconnectedness of language subsystems, but
also foster discussions on process-oriented methods of language
development, as new methodological challenges are highlighted.


\bibliographystyle{plainnat}
\bibliography{ubirata.bib}
