\chapter{How to model the influence of orthography on L2 representationswith BiPhon Neural Networks}\label{ch:silke}
\chapterauthor[1]{Silke Hamann}
\chapterauthor[1]{Chao Zhou}
\begin{affils}
\chapteraffil[1]{University of Amsterdam and University of Lisbon}
\end{affils}

\noindent 
Many studies have shown that written forms influence the acquisition of a second language. 
This influence can be helpful, as is the case of the English /æ/-/ɛ/ contrast that is notoriously
difficult for Dutch learners but where the written form can aid in the creation of the distinction
\citep{weber2004,escudero2010}.
But orthography can also cause the
creation of so-called ghost contrasts, which do not exist in the L2, as is the case with the
intervocalic singleton/geminate contrast in the L2 English of Italian speakers \citep{bassetti2017,hamann2018}.

In this talk, we illustrate how such orthographic influences on the creation of L2
representations can be formalized, by this yielding theoretical predictions that can be tested
again in experimental studies. Our formalization is performed with a symbolic neural network
based on the Bidirectional Phonetics-Phonology model \citep{boersma2007} and its extension by a
reading grammar \citep{hamann2017}.

Our main data comes from an experimental study on Mandarin \citep{zhou2020}: 23
L1-Mandarin speakers with no prior knowledge of EP (naïve listeners), representing the initial
stage of L2 acquisition, performed a delayed-imitation task. They were presented with EP
nonce words containing /ɾ/ in intervocalic onset (e.g., parafa) or word-internal coda (e.g.,
parfa), first auditorily, and then with accompanying orthography. Our results show 1) that
participants only produced L1 [ɻ] when exposed to orthography, confirming that the use of
Mandarin rhotic in L2 speech is orthographically driven; and 2) that even at the initial stage
the substitution with Mandarin [ɻ] occurs almost exclusively in coda position, reminiscent of
L2 learners \citep{zhou2017,liu2018}.

\bibliographystyle{plainnat}
\bibliography{silke.bib}


