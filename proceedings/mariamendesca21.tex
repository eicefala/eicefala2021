\chapter{A web-based open source system for speech data collection}\label{ch:mariamendesca21}
\chapterauthor[1]{Maria Mendes Cantoni}
\chapterauthor[1]{Hani Camille Yehia}
\begin{affils}
\chapteraffil[1]{Universidade Federal de Minas Gerais, Laboratório de Fonologia}
\end{affils}

Studies on speech sciences have a high demand for good quality acoustic data. Collecting data in an acoustically controlled environment, such as a laboratory, is considered the gold standard because it enhances recording quality. Nonetheless it has some important disadvantages (LABOV, 1972; LADEFOGED, 2008). Laboratories are an artificial setting that can interfere with speech behavior, limit the access to certain groups of people and hinder the random sampling of participants (HENRICH et al., 2010), besides the need for infrastructure investments. The above drawbacks could be bypassed with a procedure that allowed for remote data collection, outreaching populations away from the research center as well as researchers in centers with low financial support. A remote data collection procedure would also favor fieldwork and data collection in more natural settings, reduce displacement and infrastructure costs and contribute to the ethical principles of scientific value and fairness and justice in linguistic knowledge scope (EISENBEISS, 2014). An important component of a remote data collection system must be quality assessment. The damaging effect of ambient noise on speech recording and intelligibility is well-known (POLLACK; PICKETT, 1958; MAPP, 2008; DE DECKER, 2016). Therefore, in speech audio recordings, there is a special concern with noise level measurement and control, to ensure valid data observation and measurement through acoustical analysis. In this paper we present a web-based system designed for speech data collection. The goal is to offer a flexible tool that can be adjusted to the particular demands and materials of the study tasks: it can record and send audio files and other responses to a server and present stimuli in a variety of forms (slides, text, audio, images). The participant can either be identified or kept anonymous. An implementation of the system developed with Shiny (WINSTON et al., 2021) will be presented, with some layouts fit to data collection in speech production studies. The system will be released as an open source and freely distributed application.

\vspace{1cm}

References

DE DECKER, P. An evaluation of noise on LPC-based vowel formant estimates: Implications for sociolinguistic data collection. Linguistics Vanguard, v. 2, n. 1, 2016, p. 83-101. 

EISENBEISS, S. Extending Experimental Linguistics to Under-Researched Languages and Populations - The Principle of Justice and New Ethical Challenges. In: Second International Convention on Ethics in Research on Human Participants: Evolving norms and guidelines for the Indian context. New Delhi, 2014.

HENRICH, J.; HEINE, S. J.; NORENZAYAN, A. The weirdest people in the world? Behavioral and Brain Sciences. 2010, p. 1-75.

LABOV, W. Sociolinguistic Patterns. Oxford: Blackwell, 1972.

LADEFOGED, P. Phonetic Data Analysis: An Introduction to Fieldwork and Instrumental Techniques. Wiley-Blackwell, 2003.

MAPP, P. Designing for speech intelligibility. In: Handbook for Sound Engineers. Burlington: Elsevier, 2008. p. 1385-1412.

POLLACK, I; PICKETT, J. M. Masking of speech by noise at high sound levels. Journal of Acoustical Society of America, v. 39, 1958, p. 127-130. 

WINSTON, C.; CHENG, J.; ALLAIRE, J. J.; SIEVERT, C.; SCHLOERKE, B.; XIE, Y.; ALLEN, J.; MCPHERSON, J., DIPERT, A.; BORGES, B. shiny: Web Application Framework for R. R package version 1.6.0. 2021. https://CRAN.R-project.org/package=shiny
