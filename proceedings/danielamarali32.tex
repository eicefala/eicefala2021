\chapter{An Exemplar Model in L2 Phonology: a combined approach}\label{ch:danielamarali32}
\chapterauthor[1]{Thaïs Cristófaro-Silva}
\chapterauthor[1]{Daniela Mara Lima Oliveira Guimarães}
\begin{affils}
\chapteraffil[1]{Universidade Federal de Minas Gerais}
\end{affils}

Exemplar Models (EM) have contributed to empirical investigations in Phonology in recent years (Johnson 1997, Pierrehumbert 2001, Bybee 2001). By assuming that phonetic detail is part of phonological representations EM test the gradient implementation of phonetically motivated sound changes. Many contemporary studies point out the challenges  to connect EM approaches to explain L2 learning. This paper intends to explain L2 learning processes by combining EM and the revised Speech Learning Model (SLM-r) proposed by Flege and Bohn (2021). Both models share the assumption that perception and production interact in a dynamic fashion to construct abstract representations. In this paper we add a level of orthography, assuming that mental representations have an impact on L1 and L2 phonological and spelling systems. We claim that L2 acquisition is implemented in a gradient manner where speech production, perception, and orthography operate in an integrated system that models language as a dynamic system. We refer to this proposal as EMPL2: Exemplar Model in L2 Phonology. The major contribution of EMPL2 is to add to Exemplar Models and SLM-r the relevance of detailed phonetic information in shaping L2 phonological representations, besides connecting orthography to linguistic representations. We argue that L2 learners must be offered detailed information about pronunciation and also about the sound-spelling mapping. We hope this combined purpose contributes to better understanding of L2 phonological acquisition, leading to new approaches of teaching pronunciation.
