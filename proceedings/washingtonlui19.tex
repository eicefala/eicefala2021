\chapter{Identification of Transfer Phenomena Between Brazilian Portuguese and English as Foreign Language in Simulated Context Using Self-organizing Maps}\label{ch:washingtonlui19}
\chapterauthor[1]Washington Luis Pinho Rodrigues Filho{}
\chapterauthor[1]{João Mário de Santana Bezerra}
\chapterauthor[1]{Aratuza Rodrigues Silva Rocha}
\chapterauthor[1]{Wilson Júnior de Araújo Carvalho}
\chapterauthor[1]{Ronaldo Mangueira Lima Júnior}
\chapterauthor[1]{Fábio Rocha Barbosa}
\begin{affils}
\chapteraffil[1]{Universidade Federal do Piauí}
\end{affils}

Studies related to L1-L2 phonological transfer phenomena committed by non-native speakers gained attention in recent decades, generating new possibilities in the scientific research scenario. These phenomena are often identified and classified through a slow and laborious process of manual transcriptions. The present work proposes new possibilities for automatic identification of transfer phenomena produced during the reading process of Brazilian students learning English. The algorithm is centered on an unsupervised Self-Organizing Maps (SOM) Artificial Neural Network trained to automatically identify the transfer processes between Brazilian Portuguese and English as a Foreign Language. The audio samples used to train the algorithm were synthetically generated by the Google Translate TTS system using words in context  Five transfer processes were selected from the existing literature to demonstrate the initial results: a) the deletion of [h] in words beginning with <h> (H-deletion); b) the deletion of [h] with a change of [aj] to [i] in words beginning with <hy> (HY-i); c) changing [aj] to [i] while keeping the pronunciation of [h] in words beginning with <hy> (HY-hi); d) pronouncing silent <k> with the insertion of an epenthetic [i] in words beginning with <kn> (KN-ki); and e) d) pronouncing silent <k> without the insertion of an epenthetic [i] in words beginning with <kn> (KN-k).  To train the SOM, we use the average of the first two Formant Frequencies as signal descriptors.  The algorithm achieved identification accuracy of 58.43\% for the H-deletion process, 87.63\% for the combined HY-i and HY-hi phenomena, and 69.12\% for the combined KN-ki and KN-k process.  The non-supervised nature of the algorithm provides a new perspective for the development of a computer-assisted pronunciation training (CAPT) software capable of adaptation and high performance in the identification of these types of transfer processes.

