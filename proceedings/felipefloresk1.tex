\chapter{The social-semiotic landscapes of immigration: attrited L1 VOT production as evidence of phonological adaptability}\label{ch:felipefloresk1}
\chapterauthor[1]{Felipe Flores Kupske}
\begin{affils}
\chapteraffil[1]{Universidade Federal da Bahia, Laboratório de Ciências da Fala}
\end{affils}

In this work, I intend to propose alternative angles to observe phonological data, by tensioning perspectives of language that distance grammar from the social-semiotic landscapes and language use. Drawing from a dynamic approach to language, I discuss the role of integration into L2 settings in L1 attrition. In order to do so, I analyze the production of VOT of Brazilian Portuguese (short lag) by Brazilian immigrants in the United Kingdom, bilinguals in contact with the longer VOT pattern of English (long lag). Data from 28 Brazilian women was considered: 10 BP controls, and 18 unmarried first-generation L2 proficient immigrants (AOA>19 years old). Participants were aged between 20 and 43 years (M=26.9 years, SD=7.3 years) and resided in London for different periods of time. Immigrants were divided into groups according to length of residence (LOR) and integration into the host country. BP items with word-initial voiceless stops were considered. The measure for integration followed Yilmaz and Schmid (2015) and was based in an in-depth sociolinguistic questionnaire. The questionnaire revealed that integrated immigrants presented wider social networks in the host community and higher Full Time Equivalents (FTE) of L2 input - LOR x percentage L2 use (FLEGE; BOHN, 2021). The acoustic analysis revealed that immigrants integrated into the dominant context - with higher FTEs - yielded higher VOT values for BP-L1. The L2 long lag pattern seems to be driving the production of BP-L1 stops by integrated Brazilians who lived in London for more than four years, and L1 attrition is statistically confirmed. The same does not happen with non-integrated immigrants, who showed short lag VOT values regardless of LOR. The data indicates that speakers with higher integration rates reveal a higher tendency to adopt the patterns of the context. It is through participation in an integrative landscape that individuals would produce phonological innovations regarding the L1 VOT pattern. In addition, the data is in line with Flege and Bohn (2021), who point out that LOR in L2 settings is an imprecise index of the quantity of input bilinguals receive. LOR seems to be a valid index for immigrants who have had both the willingness and the opportunity to use the L2, factors that interact with integration. This study shows that integrated immigrants have more opportunities to accommodate speech to the L2 social landscape. Higher integration rates reveal greater social networks with L2 speakers and FTEs. Consequently, L2 dominance and L1 attrition are expected. The results show that even "adult" grammars are adaptive and sensitive to environmental changes, and indicate that non-linguistic variables, generally ignored, affect speech practices in L2 contexts, which interact with L1 attrition.

