\chapter{Some questions on L2 speech as related to colonialism}\label{ch:eleonora}
\chapterauthor[1]{Eleonora Albano}
\chapterauthor[1]{Antonio Pessotti}
\chapterauthor[1]{Carla Diaz}
\begin{affils}
\chapteraffil[1]{LAFAPE-IEL-UNICAMP \& CNPq}
\end{affils}

\noindent
The first aim of this talk is to revisit the question of phonetic drift in $L_2$
speechin light of new data and theory. The new data consist of a sizeable set
of acousticphonetic measures of the speech of Quechua and Spanish monolinguals
and bilingualsresiding in Peru. The theoretical innovation draws on two
sources: the relativelyfamiliar concept of accommodation, introduced by \citet{giles1973}
insociolinguistics, and the less familiar concept of
coloniality, introduced by \citep{quijano2000} in sociology. At the same time, it
aims at showing that phonetic analysis basedon gestural phonology can open new
avenues for exploring the relationship betweenthese two concepts as \textit{explanans}
for $L_2$ pronunciation in an ethnically diverseenvironment.

Accommodation refers to a “constant movement toward and away from others, by
changing one’s communicative behavior” \citep{giles2007}. It encompasses
speech and various other communicative behaviors. Moreover, it has a convergent
side -- enhancing similarities between interlocutors -- and a divergent one --
enhancing differences between interlocutors. Both can occur between two or more
people or within and across speech communities. Some acoustic phonetic
parameters have been useful to tap such shifts \citep[e.g., VOT, as in the pioneering work of][]{sancier1997}.

The concept of coloniality refers to “how colonial patterns of power and
inequality exceed the spatial and temporal boundaries of empire and colony”
\citep{roche2019}. It aims at dealing with the epistemology underlying the
pervasive replication of colonial social, economic, and cultural practices in
postcolonial societies \citep{quijano2000}.

We will start by revisiting earlier work on phonetic drift in L2 conducted at
our lab -- \textit{Laboratório de Fonética e Psicolinguística} (LAFAPE). \citet{ramirez2013}
showed that contact situations may exhibit intralinguistic phonetic
drift in both $L_1$ and $L_2$. In turn, \citet{albano2020} reported preliminary
observations of intralinguistic drift attributable to language attrition in
Quechua/Spanish bilinguals residing in Brazil.

We believe that the understanding of the results of both of these works can be
considerably improved by reference to the above-defined concepts. In
particular, some intriguing signs of partial loss of Quechua stop distinctions
shown by the expatriated Peruvians can be interpreted as mistiming of
articulatory gestures converging towards those of the two hegemonic languages
(namely, Spanish and Portuguese).

Then we will move on to inquire how the study conducted in Peru can elucidate
our questions about Quechua/Spanish relations. All data collection on this
topic was part of Carla Diaz's requisites for completing her bachelor and
master’s degrees in linguistics \citep{diaz2018,diaz2021}. 

Carla recorded 10 Spanish monolinguals and 10 Quechua/Spanish bilinguals in
Lima in August 2019. Then she travelled to Cuzco to record 11 monolingual
Quechua speakers, with the help of a bilingual friend specializing in Quechuan
literature.

The corpus, similar to that of \citet{albano2020}, focused on Quechua and
Spanish stop contrasts. The analysis, likewise, employed measurements that have
been used in the description of Quechua: VOT, amplitude of the stop burst, $f_0$,
and $H1-H2$.

The results show that, unlike the residents of Brazil, the residents of Lima
have no trouble distinguishing stops within the Quechua series or
differentiating them from Spanish stops. The remarkable fact is that divergence
from Spanish was more frequent than convergence. Moreover, certain distinctions
were enhanced by shifting the acoustic parameters beyond the values of the
monolingual group.

After Carla defends her thesis, we are planning to conduct finer-grained
analyses considering the linguistic values and attitudes captured by our
sociolinguistic questionnaire. May we succeed in helping unravel the
coloniality issues behind the subtle attempts of Peruvian Quechua speakers at
resisting diglossia and language loss.

\bibliographystyle{plainnat}
\bibliography{eleonora.bib}

