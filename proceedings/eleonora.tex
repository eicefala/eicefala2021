\chapter{Some questions on L2 speech as related to colonialism}\label{ch:eleonora}
\chapterauthor[1]{Eleonora Albano}
\chapterauthor[1]{Antonio Pessotti}
\chapterauthor[1]{Carla Diaz}
\begin{affils}
\chapteraffil[1]{LAFAPE-IEL-UNICAMP \& CNPq}
\end{affils}

\noindent
The first aim of this talk is to revisit the question of phonetic drift in $L_2$
speechin light of new data and theory. The new data consist of a sizeable set
of acousticphonetic measures of the speech of Quechua and Spanish monolinguals
and bilingualsresiding in Peru. The theoretical innovation draws on two
sources: the relativelyfamiliar concept of accommodation, introduced by \citeauthor{Giles1973}
insociolinguistics, and the less familiar concept of
coloniality, introduced by Quijano(2000) in sociology. At the same time, it
aims at showing that phonetic analysis basedon gestural phonology can open new
avenues for exploring the relationship betweenthese two concepts as \textit{explanans}
for $L_2$ pronunciation in an ethnically diverseenvironment.

Accommodation refers to a “constant movement toward and away from others,by
changing one’s communicative behavior” (GILES \& OGAY, 2007). It
encompassesspeech and various other communicative behaviors. Moreover, it has a
convergent side -- enhancing similarities between interlocutors -- and a divergent
one -- enhancingdifferences between interlocutors. Both can occur between two or
more people or withinand across speech communities. Some acoustic phonetic
parameters have been useful totap such shifts (e.g., VOT, as in the pioneering
work of SANCIER \& FOWLER, 1997).

The concept of coloniality refers to “how colonial patterns of power
andinequality exceed the spatial and temporal boundaries of empire and colony”
(ROCHE,2019). It aims at dealing with the epistemology underlying the pervasive
replication ofcolonial social, economic, and cultural practices in postcolonial
societies (QUIJANO,2000).
 
We will start by revisiting earlier work on phonetic drift in $L_2$ conducted at
ourlab -- \textit{Laboratório de Fonética e Psicolinguística} (LAFAPE). Ramirez
et al. (2011)showed that contact situations may exhibit intralinguistic
phonetic drift in both $L_1$ and $L_2$. In turn, Albano et al. (2020) reported
preliminary observations of intralinguistic driftattributable to language
attrition in Quechua/Spanish bilinguals residing in Brazil.

We believe that the understanding of the results of both of these works can
beconsiderably improved by reference to the above-defined concepts. In
particular, someintriguing signs of partial loss of Quechua stop distinctions
shown by the expatriatedPeruvians can be interpreted as mistiming of
articulatory gestures converging towardsthose of the two hegemonic languages
(namely, Spanish and Portuguese).

Then we will move on to inquire how the study conducted in Peru can
elucidateour questions about Quechua/Spanish relations. All data collection on
this topic waspart of Carla Diaz's requisites for completing her bachelor and
master’s degrees inlinguistics (DIAZ, 2018; 2021).

Carla recorded 10 Spanish monolinguals and 10 Quechua/Spanish bilinguals inLima
in August 2019. Then she travelled to Cuzco to record 11 monolingual
Quechuaspeakers, with the help of a bilingual friend specializing in Quechuan
literature.

The corpus, similar to that of Albano et al. (2020), focused on Quechua
andSpanish stop contrasts. The analysis, likewise, employed measurements that
have beenused in the description of Quechua: VOT, amplitude of the stop burst,
$f_0$, and $H1-H2$.

The results show that, unlike the residents of Brazil, the residents of Lima
haveno trouble distinguishing stops within the Quechua series or
differentiating them fromSpanish stops. The remarkable fact is that divergence
from Spanish was more frequentthan convergence. Moreover, certain distinctions
were enhanced by shifting the acousticparameters beyond the values of the
monolingual group.

After Carla defends her thesis, we are planning to conduct finer-grained
analysesconsidering the linguistic values and attitudes captured by our
sociolinguisticquestionnaire. May we succeed in helping unravel the coloniality
issues behind thesubtle attempts of Peruvian Quechua speakers at resisting
diglossia and language loss.

\bibliographystyle{plainnat}
\bibliography{eleonora.bib}

