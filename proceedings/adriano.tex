\chapter{Using statistical learning techniques to determine Cantonese lexical tones from the acoustic and visual components of speech}\label{ch:adriano}
\chapterauthor[1]{João Vitor Possamai de Menezes}
\chapterauthor[1]{Hani Camille Yehia}
\chapterauthor[1]{Adriano Vilela Barbosa}
\begin{affils}
\chapteraffil[1]{Universidade Federal de Minas Gerais}
\end{affils}

\noindent 
This mini-course presents an introduction to the use of statistical learning techniques to speech
processing problems. More specifically, we show how classification techniques can be used to
predict lexical tones in Cantonese from the associated measurements of both the acoustic and the
visual (to a lesser degree) components of speech. The acoustic and visual data we use were recorded
during a speech production experiment where a native speaker of Cantonese produced a set of
words spanning the full range of Cantonese tones. The visual data consists of 3D trajectories of
markers on the subject’s face and head recorded with an Optotrak. The acoustic component is
represented by F0 trajectories extracted from the speech acoustics. The idea is to use the F0 and
marker trajectories as input vectors to train classifiers to predict the lexical tones. However, these
trajectories cannot be used directly because they have different durations for different tokens
(utterances), whereas all input vectors to the classifiers must have the same dimension. In order to
make all input vectors the same length, regardless of the duration of the utterances, all trajectories
(both F0 and markers) are approximated by polynomials of a given order and represented by the
corresponding coefficients. The polynomial coefficients are then used as input vectors to train
different classification models (LDA, SVM, K-nearest neighbors, etc). The performance of the
models is estimated by means of k-fold cross validation. Although the statistical learning techniques
we present are applied to a specific problem (estimating Cantonese lexical tones from the acoustical
and visual components of speech), they are general and can be equally applied to a wide range of
problems. All procedures presented in the mini-course are developed in the R language.
