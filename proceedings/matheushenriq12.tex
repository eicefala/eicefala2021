\chapter{Preliminary analysis of coda tap produced by Brazilian Portuguese speakers from a city in Santa Catarina}\label{ch:matheushenriq12}
\chapterauthor[1]{Matheus Henrique Germano}
\chapterauthor[1]{Adelaide Hercília Pescatori Silva}
\begin{affils}
\chapteraffil[1]{Federal University of Paraná}
\end{affils}

Studies indicate the occurrence of a short vowel-like segment after coda tap in Brazilian Portuguese (BP), as in [po.'deɾ] (SILVA, 1996, CLEMENTE, 2005). A similar phenomenon has been described preceding tap in consonant clusters, as in ['gɾu.pʊ] (SILVA, 1996, NISHIDA, 2005). According to Clemente (2005), the acoustic characteristics of the vowel-like segment that follows coda tap are similar to those of schwa. On the other hand, Nishida (2005) found that the acoustic characteristics of the vowel-like segment that precede tap in clusters are similar to that of the nuclear vowel of the syllable. The same characteristics were described for tap in word-initial position (NISHIDA, 2009). These results led Nishida (2009) to claim that tap is always produced between two vocalic segments. However, phonetic literature does not present data on word-medial coda tap, e.g. [ˈnɔɾ.t͡ʃɪ]. Bilharva da Silva (2019) was the first to acoustically analyze taps in this position based on data from the control group of a sociolinguistic study. In light of this, the objective of this study is to acoustically analyze the BP tap in both word-medial and final coda positions. To meet this objective, we collected data from three native speakers of BP living in Timbó, Santa Catarina, Brazil. This city was selected based on our auditory impression and a paper by Brescancini and Monaretto (2008). According to the authors, tap is the most frequent rhotic variant in syllable-final position in the speech of people from Blumenau, a city near Timbó. The data was acoustically analyzed using Praat (BOERSMA; WEENINK, 2021) and the results indicated that the participants from Timbó produced tap in coda position in most data. Furthermore, the vowel preceding the tap presented a lowering of F3. Subsequently, we developed a new experiment to collect more data. Four native speakers of BP from Timbó participated in this experiment and the data has yet to be analyzed. We will measure the first three formants of the nuclear vowel and vowel-like segment considering their middle, in order to compare them. However, a preliminary analysis of these data showed not only a lowering of F3, but also a raising of F2, from the middle to the final point of the vowel preceding coda tap. The formant values at the end of the nuclear vowel are similar to those of the vowel-like segment. Thus, a measurement of the final-point of the vowel may also be necessary. It is worth mentioning that there are no studies on the speech of people from Timbó, either from a phonetic or sociolinguistic perspective. Therefore, the present study may contribute to both perspectives.

\vspace{1ex}

References

BILHARVA DA SILVA, Felipe. O contato português-pomerano na produção dos grupos [Cɾ] e [ɾC]: o caso das vogais suarabácticas. 2019. 281 f. Tese (Doutorado) - Curso de Letras, Pontifícia Universidade Católica do Rio Grande do Sul, Porto Alegre, 2019. Disponível em: https://repositorio.pucrs.br/dspace/handle/10923/15831. Acesso em: 08 fev. 2021.

BOERSMA, Paul; WEENINK, David. Praat: doing phonetics by computer. Versão 6.1.50. 20 jun. 2021. Disponível em: https://www.fon.hum.uva.nl/praat/ Acesso em: 05 jul. 2021.

BRESCANCINI, Cláudia; MONARETTO, Valéria Neto de Oliveira. Os róticos no sul do Brasil: panorama e generalizações. Signum: Estudos da Linguagem, Londrina, v. 11, n. 2, p.51-66, 2008. Disponível em: https://lume.ufrgs.br/bitstream/handle/10183/183250/000701231.pdf. Acesso em: 18 fev. 2021.

CLEMENTE, Felipe Costa. Análise Acústica do Tap em Coda no Português Brasileiro. 2005. 52 f. Monografia - Curso de Letras, Universidade Federal do Paraná, Curitiba, 2005.

NISHIDA, Gustavo. Análise acústica do tap em grupos do PB. 2005. 59 f. Monografia - Curso de Letras, Universidade Federal do Paraná, Curitiba, 2005.

SILVA, Adelaide Hercilia Pescatori. Para a descrição fonetico-acustica das liquidas no portugues brasileiro: dados de um informante paulistano. 1996. 230p. Dissertação (mestrado) - Universidade Estadual de Campinas, Instituto de Estudos da Linguagem, Campinas, SP. Disponível em: <http://www.repositorio.unicamp.br/handle/REPOSIP/270650>. Acesso em: 04 dez. 2020.
