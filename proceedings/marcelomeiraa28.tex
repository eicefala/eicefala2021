\chapter{A pilot study on the prosodic aspects of child-directed signing in Brazilian sign language}\label{ch:marcelomeiraa28}
\chapterauthor[1]{Marcelo Meira Alves}
\chapterauthor[1]{Maria de Fátima de Almeida Baia}
\chapterauthor[1]{Adriana Stella Cardoso}
\begin{affils}
\chapteraffil[1]{Universidade Estadual do Sudoeste da Bahia}
\end{affils}

%%%%%%%%%%%%%%%%%%%%%%%%%%%%%%%%%%%%%%%%%%%%%%%%%%%%%%%%%%%%%%%%%%%%%%

This pilot study investigates the child-directed signing (CDSig) in Brazilian sign language (Libras). To understand the phenomenon CDSig, our research is based on studies on oral languages, such as Elliot (1982), Ferreira (1990),  Fernald (1989), Kuhl (1997), Cavalcanti (1999 ) and Ferreira, Baia and Pacheco  (2019), in which speech alterations are observed at syntactic, discursive, lexical and prosodic levels. Furthermore, regarding the prosodic aspects inherent to CDSig, we support our work on the studies by Holzrichter and Meier (2000) and Fuks (2019) on Israeli and Hebrew sign languages, These studies show that there are phonetic modifications in CDSig, such as displacement, repetitions, stretching and amplification of signs (HOLZRICHTER; MEIER, 2000; FUKS, 2019) as well as the intensification of iconic signs in order to facilitate the communication with their babies in the early stages. We assume the hypothesis that as in oral languages the phenomenon also occurs in Libras as it is a natural human language. To verify the hypothesis, a picture naming experiment was designed with a list of 51 words based on previous studies on child-directed speech (FERGUSON, 1964; STOEL-GAMMON, 1976; CLARK, 2005; BAIA, 2010). The results indicated the occurrence of changes in the phonetic aspects of the sign, such as changes in the movements of the arms and hands, changes in the configuration of the hand, reconfiguration of non-manual expressions, body expressions and iconicity. (Scholarship: FAPESB - 072.4195.2020.0010174-4)
