\chapter{Experiments on speech analysis in near-ultrasound using conventional audio equipment}\label{ch:melchioraugus10}
\chapterauthor[1]{Melchior Augusto Syrio de Melo}
\chapterauthor[1]{Hani Camille Yehia}
\begin{affils}
\chapteraffil[1]{Universidade Federal de Minas Gerais, CEFALA}
\end{affils}

The advancements in speech recognition technology and its broad availability in smart devices as Virtual Assistants (VAs) has raised concerns about the security of these systems and how these human-machine interfaces might be used to access private information or to execute actions on behalf of the device's owner without his or her knowledge. The awareness of dangerous permissions requested by VAs motivated studies to better understand and to test potential attacks on devices. These efforts resulted in the discovery of several flaws in VAs - including attacks using near ultrasound, of which the most popular is known as Dolphin Attack. Its execution is due to two factors: a flaw in the device microphone and the design of audio equipment. Conventional microphones and audio tweeters allow a researcher to take advantage of the slow decay in the near ultrasound frequency response in order to generate and record ultrasound signals. The deformation in the membrane of microphones, which causes distortions, allows an especially crafted radio-like signal containing speech information to be translated in the frequency spectrum and to be successfully interpreted by a speech recognition service. This paper shows the mechanisms that allow a speech recording transmitted in a near ultrasound acoustic wave to be recognized and executed in a device, surpassing its software and hardware restrictions. It will also present experiments concerning the high-end frequency response of conventional audio equipment systems, composed by condenser microphones, several smartphone microphones, speakers and an audio interface.
