\chapter{Online speech data collection: Whether and how researchers should do it}\label{ch:cesartelo33}
\chapterauthor[1]{Cesar Teló}
\begin{affils}
\chapteraffil[1]{Universidade Federal de Santa Catarina}
\end{affils}

With the onset of the COVID-19 pandemic, researchers across the globe were forced to interrupt in-person data collection and find new ways to do research. While many turned to their previously collected data or started doing secondary research, others began to explore and adopt remote, online tools for data collection. Collecting speech data outside a laboratory equipped with sound-proof booths and professional microphones for subsequent acoustic analysis (Zhang et al., 2021) or perceptual purposes (Crowther \& Urada, 2021) is, however, a debatable topic among speech scientists. Sound quality, however, is not the only challenge faced by those who record participants remotely during a pandemic: When looking for online platforms and software to run their experiments, researchers may be overwhelmed by the number of resources available, their assets, and their limitations.
Therefore, the objective of the present research is twofold: (1) To assess the state of the art in online speech data collection and its implications to acoustic analyses and perceptual experiments; and (2) to present a summary and review of the main platforms that may be used to collect speech data remotely. This presentation results from an extensive survey conducted at the Psycholinguistics Lab at the Language Research Centre of the University of Calgary and at the NUPFFALE Research Group at Universidade Federal de Santa Catarina. Its purpose is to aid researchers who conduct studies on L1 and/or L2 speech decide whether and how to collect speech data online based on recent findings and platforms currently available.

Crowther, D., Urada, K. (2021, June 18-19). Face-to-face versus online second language speech elicitation: Listeners’ perceptions of audio quality [Paper presentation]. 12th Pronunciation in Second Language Learning and Teaching, St. Catherines, Ontario, Canada.

Zhang, C., Jepson, K., Lohfink, G., \& Arvaniti, A. (2021). Comparing acoustic analyses of speech data collected remotely. The Journal of the Acoustical Society of America, 149(6), 3910–3916. https://doi.org/10.1121/10.0005132

