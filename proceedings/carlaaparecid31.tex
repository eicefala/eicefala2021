\chapter{Fatigue and sleepiness in aviators: analysis of voice, speech and language variations}\label{ch:carlaaparecid31}
\chapterauthor[1]{Carla Aparecida de Vasconcelos}
\chapterauthor[1]{Maurílio Nunes Vieira}
\chapterauthor[1]{Hani Camille Yehia}
\begin{affils}
\chapteraffil[1]{Universidade Federal de Minas Gerais}
\end{affils}


Fatigue can, from a behavioral perspective, be defined as the state of
performance impairment following a period of mental or physical effort \citep{akerstedt1990,akerstedt2008,banks2013}.
Sleepiness refers to the transition between wakefulness and sleep and is
defined as the tendency to fall asleep even when the individual is supposed to
be active, for example, performing work. However, both fatigue and sleepiness
are characterized by decreased ability to work, to operate machinery safely,
attention failures, cognitive slowing, and memory problems \citep{akerstedt2008,lim2010}.

Central human fatigue and sleepiness have aroused interest in aviation area
worldwide. This is due to the number of accidents and the expressive
involvement of human factors among the causes \citep{banks2013,greeley2007,greeley2013,krajewski2007,krajewski2009,krajewski2010}.

In Brazil, according to CENIPA (Center for Research and Prevention of
Aeronautical Accidents), the rate is 1 accident every 2 days and 90\% are caused
by human factors.

According to NASA (National Aeronautics and Space Administration),
fatigue/sleepiness would contribute to approximately 20\% of air crashes in the
world. But despite the risks that fatigue and sleepiness add to the safety,
only 19 Brazilian aeronautical occurrences presented them as contributing
factors. This is due to the absence of a methodology for detecting these
signals/symptoms.

In this sense, the objective of this study was to develop a method for
detecting aviators human fatigue and sleepiness based on acoustic correlates of
voice, speech and language. To this end, this research was subdivided into 5
substudies.

In the first substudy, speech samples from pilots complaining/suspected of
fatigue/sleepiness were compared to a control group. The results were also
compared to the Fatigue Avoidance Scheduling Tool (FAST) \citep{fastfatigue}. 

From the second to the fourth substudy, the researchers analyzed 3 real cases
of accident, with evidence of fatigue/sleepiness as contributing factors in 2
of these, and speech samples of pilots recorded before the accident were
compared with those recorded during the crash. 

In the fifth substudy, the aviators were screened through four
fatigue/sleepiness scales (Karolinska Sleepiness Scale -- KSS
\citep{akerstedt1990}; Epworth Sleepiness Scale -- ESS \citep{johns1993},
Samn-Perelli Fatigue Scale -- SPFS \citep{samn1982} and Yoshitake Fatigue Scale --
YFS \citep{yoshitake1971}) and speech evaluation was performed in two
situations: on a day off when they were not complaining about
sleepiness/fatigue and during a working day in which they were fatigued/sleepy. 

The data from the scales were statistically analyzed using the Friedman test
(KSS and SPFS) and Wilcoxon test (ESS and YFS). It was observed that fatigue
and sleepiness increased on the working day.

For speech analysis in the fifth substudy, the paired GLM (General Linear
Model) was used. The entire audio was manually segmented in PRAAT for the
analysis of the temporal organization of speech \citep{grosjean1978,maclay1959} and nine variables were
extracted from speech: elocution rate, mean pause duration, total pause rate,
fluent pause rate, disfluent pause rate, disfluent silent pause rate, disfluent
filled pause rate, articulation rate and total silent pause rate. The first
seven showed significant variation over time, when participants showed
increased fatigue and sleepiness indexes.

In addition, PCA (Principal Component Analysis) was applied to reduce the
extracted variables to four. It was also found that it is possible to use
Linear Discriminant Analysis (LDA) to group individuals and classify new cases
(with or without fatigue and sleepiness) based on a database built for this
purpose.

There was quantitative and qualitative variation of voice, speech and language
in the 2 out of the 3 cases where the accident occurred in the presence of
signals of fatigue/sleepiness.

In the first substudy, statistical and qualitative variations were also
observed between the control group and the group with complaints.

Through these studies, we found that the acoustic and perceptive parameters of
voice, speech and language analyzed here are sufficiently robust to detect
central fatigue and sleepiness.


\bibliographystyle{plainnat}
\bibliography{carlaaparecid31.bib}
