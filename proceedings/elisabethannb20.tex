\chapter{The effect of perception training with synthetic and natural stimuli on BP learners` ability to identify the English vowels /æ-ɛ/}\label{ch:elisabethannb20}
\chapterauthor[1]{Elisabeth Ann Bunch Oliveira da Rosa}
\begin{affils}
\chapteraffil[1]{Universidade Federal de Santa Catarina}
\end{affils}

Perception is a crucial component in the acquisition of a second language (L2) regarding oral communication. Research has revealed factors that can often predict the specific difficulties for acquiring certain sounds in the L2. Vowels tend to pose a special difficulty, with the English vowel pair /æ-ɛ/ being particularly difficult for native Brazilian Portuguese (BP) learners, who may not distinguish the two as separate, but instead perceive them both as the vowel /ɛ/. Perception training with synthetic stimuli is one way to assist L2 learners in the formation of new vowel categories. Synthetic stimuli that controls for vowel duration may be especially effective for this type of training, as it assists learners in developing appropriate cue weighting ability for the L2. Based on the discussion above, the present study investigates the effectiveness of synthetic versus natural stimuli for perception training on the ability of BP learners of English to distinguish the vowel pair /æ-ɛ/. The participants for this study will be 61 Brazilian learners of English, divided into natural stimuli, synthetic stimuli, and control groups. Participants will receive perception training using the online research platform Gorilla and complete pre- and post-training perception tests to measure their progress (or lack thereof).

\vspace{1ex}

BEST, C.; TYLER, M. Nonnative and second-language speech perception:
Commonalities and complementarities. In: MUNRO, Murray; BOHN, Ocke-Schwen (ed.). Language experience in second language speech learning: In honor of James Emil Flege. [S. l.: s. n.], 2007. cap. 2, p. 13-34. Disponível em: https://www.researchgate.net/publication/258209938\_ Nonnative\_and\_second-language\_speech\_perception\_Commonalities \_and\_complementar ities Acesso em: 21 out. 2019.

CARLET, A; CEBRIAN, J. (2014) Training Catalan speakers to identify L2
consonants and vowels: A short-term high variability training study. Concordia Working Papers in Applied Linguistics, 5.

CHENG, B.; ZHANG, X.; FAN, S.; ZHANG, Y. (2019) The role of temporal
acoustic exaggeration in high variability phonetic training: A behavioral and ERP study. Frontiers in Psychology, 10, (1178). Doi: 10.3389/fpsyg.2019.01178.

COLANTONI, L.; STEELE, J.; ESCUDERO, P. Second language speech:
Theory and practice. Cambridge: Cambridge University Press, 2015.

ESCUDERO, P.; BOERSMA, P. (2004). Bridging the gap between L2 speech
perception research and phonological theory. Studies in Second Language Acquisition. Doi: 10.1017/S027226310404002

FLEGE, James. Assessing constraints on second-language segmental
production and perception. In: A., Meyer; N., Schiller (ed.). Phonetics and phonology in language comprehension and production: Differences and similarities. Berlin: Mouton de Gruyter, 2003. p. 319-355. Disponível em: http://jimflege.com/chapters\_after\_2000.html. Acesso em: 21 out. 2019

LADEFOGED, Peter. Vowels and consonants. 2. ed. Oxford, UK: Blackwell
Publishing, 2007.

LIMA JR., R. (2017) The influence of metalinguistic knowledge of segmental
phonology on the production of English vowels by Brazilian undergraduate students. Ilha do Desterro, 70, 117-130.

NOBRE-OLIVEIRA, D. (2007) The effect of perceptual training on the learning
of English vowels by Brazilian Portuguese speakers. Florianópolis, SC. Dissertação de Doutourado. Universidade Federal de Santa Catarina.

RATO, A. (2014) Effects of perceptual training on the identification of English
vowels by native speakers of European Portuguese. Concordia Working Papers in Applied Linguistics, 5.

RAUBER, A. S. (2006). Perception and production of English vowels by
Brazilian EFL speakers. Florianópolis, SC. Dissertação de Doutourado. Universidade Federal de Santa Catarina.



